
            %--------------------------%
\chapter {~~{Frequently asked questions}}\label{\numb section 13}
            %--------------------------%

          %------------------------------------------%
\section{~~How do I compile my own {\tt main} files ?}\label{\numb section 13.\numb parag 1}
          %------------------------------------------%

See paragraph \ref{\numb section 11.\numb parag 16}.


          %-----------------------------------------------------------------%
\section{~~Is {\maniFEM} thread-safe ? Does it support parallel processing ?}
          %-----------------------------------------------------------------%
\label{\numb section 13.\numb parag 2}

No.


          %-----------------------------------------------------%
\section{~~What's the difference between a mesh and a manifold ?}
          %-----------------------------------------------------%
\label{\numb section 13.\numb parag 3}

A {\small\tt\verm{Mesh}} is a collection of {\small\tt\verm{Cell}}s of the
same dimension.

A {\small\tt\verm{Manifold}} is a {\tt C++} object modelling an abstract concept.
There are Euclidian {\small\tt\verm{Manifold}}s, there are implicitly defined
sub-{\small\tt\verm{Manifold}}s, there are parametric {\small\tt\verm{Manifold}}s
and also quotient {\small\tt\verm{Manifold}}s.

A {\small\tt\verm{Mesh}} may be the discretization of a {\small\tt\verm{Manifold}}
or of a piece (region) of a {\small\tt\verm{Manifold}}.

A {\small\tt\verm{Manifold}} is smooth (unless it has singular points, like in 
paragraphs \ref{\numb section 3.\numb parag 21}, \ref{\numb section 3.\numb parag 22},
\ref{\numb section 7.\numb parag 15}, \ref{\numb section 7.\numb parag 16}),
a {\small\tt\verm{Mesh}} can be thought of as a polygonal surface.

See paragraphs \ref{\numb section 2.\numb parag 4}, \ref{\numb section 2.\numb parag 11},
\ref{\numb section 2.\numb parag 17} and \ref{\numb section 2.\numb parag 18};
see sections \ref{\numb section 7} and \ref{\numb section 8}.


          %-----------------------------------------------------%
\section{~~What's the difference between methods {\tt glue\_\,on\_\,bdry\_\,of} and
  {\tt add\_\,to\_\,mesh}, or between {\tt cut\_\,from\_\,bdry\_\,of} and
  {\tt remove\_\,from\_\,mesh} ?}
          %-----------------------------------------------------%
\label{\numb section 13.\numb parag 4}

For adding a cell to a mesh which is the boundary of some other cell,
you should use {\small\tt glue\_\,on\_\,bdry\_\,of} and {\small\tt cut\_\,from\_\,bdry\_\,of}.
Otherwise, you should use {\small\tt add\_\,to\_\,mesh} and {\small\tt remove\_\,from\_\,mesh}.

In some cases (when you are sure there is no mesh ``above'' that other cell)
it makes no difference, so you can use either.
For instance, in {\small\tt\verm{Cell}} constructors and destructors
we have chosen to use {\small\tt add\_\,to\_\,mesh} and {\small\tt remove\_\,from\_\,mesh}
because they are slightly faster.


          %----------------------------------------------------%
\section{~~What happens if we try to identify a twisted pair of
           opposite faces of a parallelogram ?}
          %----------------------------------------------------%
\label{\numb section 13.\numb parag 5}

If, in paragraph \ref{\numb section 7.\numb parag 19}, we ask {\maniFEM} to identify
{\small\tt BC} with {\small\tt DA} (forgetting to reverse {\small\tt DA}),
{\maniFEM} will complain about orientations.

\ManiFEM{} has been designed around the concept of oriented mesh so it will
bluntly refuse to mesh a non-orientable manifold.
Had it been otherwise, identifying {\small\tt BC} with {\small\tt AD}
would produce a M\"obius strip.

Similarly, in paragraph \ref{\numb section 7.\numb parag 20}, a statement like

\begin{Verbatim}[commandchars=\\\{\},formatcom=\small\tt,
   baselinestretch=0.94,framesep=2mm                     ]
   \verm{Mesh} \azul{torus} = square.fold ( \textcolor{tag}{tag}::identify, AB, \textcolor{tag}{tag}::with, CD.reverse(),
                              \textcolor{tag}{tag}::identify, BC, \textcolor{tag}{tag}::with, DA,
                              \textcolor{tag}{tag}::use_existing_vertices                 );
\end{Verbatim}

\noindent would produce a Klein surface if {\maniFEM} were not so stubbornly reliant on
oriented meshes.

On the other hand, the statement

\begin{Verbatim}[commandchars=\\\{\},formatcom=\small\tt,
   baselinestretch=0.94,framesep=2mm                     ]
   \verm{Mesh} \azul{torus} = square.fold ( \textcolor{tag}{tag}::identify, AB, \textcolor{tag}{tag}::with, CD,
                              \textcolor{tag}{tag}::identify, BC, \textcolor{tag}{tag}::with, DA,
                              \textcolor{tag}{tag}::use_existing_vertices       );
\end{Verbatim}

\noindent would produce a projective plane if {\maniFEM} were not so reluctant of
non-oriented objects.


          %---------------------------------------------------------------%
\section{~~Why do we declare coordinates to be of type Lagrange, of degree 1 ?}
          %---------------------------------------------------------------%
\label{\numb section 13.\numb parag 6}

That's because we want (positive) vertices to have coordinates.
Other cells will have no coordinates.
See paragraph \ref{\numb section 5.\numb parag 1}.


          %-----------------------------------------------------------%
\section{~~Will {\maniFEM} support parallel processing in the future ?}
          %-----------------------------------------------------------%
\label{\numb section 13.\numb parag 7}

At the moment, this is not part of our plans.


