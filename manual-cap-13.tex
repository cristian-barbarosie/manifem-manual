
            %--------------------------%
\chapter {~~{Frequently asked questions}}\label{\numb section 13}
            %--------------------------%

          %-----------------------------------------------------%
\section{~~What's the difference between a mesh and a manifold ?}
          %-----------------------------------------------------%
\label{\numb section 13.\numb parag 1}

A {\small\tt\verm{Mesh}} is a collection of {\small\tt\verm{Cell}}s of the
same dimension.

A {\small\tt\verm{Manifold}} is a {\tt C++} object modelling an abstract concept.
There are Euclidian {\small\tt\verm{Manifold}}s, there are implicitly defined
sub-{\small\tt\verm{Manifold}}s, there are parametric {\small\tt\verm{Manifold}}s
and also quotient {\small\tt\verm{Manifold}}s.

Often, a {\small\tt\verm{Mesh}} is a discretization of a {\small\tt\verm{Manifold}}.

See paragraphs \ref{\numb section 2.\numb parag 3}, \ref{\numb section 2.\numb parag 9},
\ref{\numb section 2.\numb parag 12} and \ref{\numb section 2.\numb parag 13};
see sections \ref{\numb section 7} and \ref{\numb section 8}.


          %-----------------------------------------------------%
\section{~~What's the difference between methods {\small\tt glue\_\,on\_\,bdry\_\,of} and
  {\small\tt add\_\,to\_\,mesh}, or between {\small\tt cut\_\,from\_\,bdry\_\,of} and
  {\small\tt remove\_\,from\_\,mesh} ?}
          %-----------------------------------------------------%
\label{\numb section 13.\numb parag 2}

When a mesh is the boundary of a cell, you should use {\small\tt glue\_\,on\_\,bdry\_\,of} and
{\small\tt cut\_\,from\_\,bdry\_\,of}.
Otherwise, you should use {\small\tt add\_\,to\_\,mesh} and {\small\tt remove\_\,from\_\,mesh}.

In some cases (when you are sure there is no mesh ``above'' this cell), it makes no difference,
so you can use either.
This happens for {\small\tt\verm{Cell}} constructors and destructors
(it is documented in the source code).
We have chosen to use {\small\tt add\_\,to\_\,mesh} and {\small\tt remove\_\,from\_\,mesh}
because they are slightly faster.
