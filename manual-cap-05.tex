
          %----------------------------------------------%
\chapter{~~Fields, functions and variational formulations}\label{\numb section 5}
          %----------------------------------------------%


          %--------------------%
\section{~~Fields and functions}\label{\numb section 5.\numb parag 1}
          %--------------------%

As the reader may have already noticed, all examples in {\maniFEM} begin by declaring a
Euclidian {\small\tt \verm{Manifold}} and then go on by building a coordinate system :

\begin{Verbatim}[commandchars=\\\{\},formatcom=\small\tt,
   baselinestretch=0.94,framesep=2mm                     ]
   \verm{Manifold} RR2 ( \verm{tag}::Euclid, \verm{tag}::of_dim, 2 );
   \verm{Function} \azul{xy} = RR2.build_coordinate_system ( \verm{tag}::Lagrange, \verm{tag}::of_degree, 1 );
   \verm{Function} \azul{x} = xy[0], \azul{y} = xy[1];
\end{Verbatim}

See paragraph \numb section 11.\numb parag 3 for more details about tags.

In the above, {\small\tt \verm{xy}} is a {\small\tt \verm{Function}}, a vector field actually,
with two components, {\small\tt x} and {\small\tt y}.
The declaration of {\small\tt xy} starts a complex process; under the curtains
{\maniFEM} declares a {\small\tt \verm{Field}} object associated to {\small\tt xy}.
The {\small\tt \verm{Field}} object changes the behaviour of {\maniFEM} in what regards
initialization of {\small\tt \verm{Cell}}s.
Since {\small\tt xy} is of type Lagrange of degree 1, each newly built vertex
{\small\tt \verm{Cell}} will have memory space reserved for two {\small\tt double}
precision numbers.
If {\small\tt A} is a vertex {\small\tt \verm{Cell}}, an assignment like {\small\tt x(A) = 1.5}
sets the value of the {\small\tt x} component of the {\small\tt \verm{Field}} associated to
{\small\tt xy} at {\small\tt A}.

If we declare {\small\tt xy} to be of type Lagrange of degree 2, not only future vertices will
have space reserved for two {\small\tt double}s, but also future segments.
So, we may assign the value of {\small\tt y} at the middle of segment {\small\tt AB} by using
the syntax {\small\tt y(AB) = 0.75}.

{\small\tt \verm{Function}} objects allow for arithmetic expressions like in

\begin{Verbatim}[commandchars=\\\{\},formatcom=\small\tt,
   baselinestretch=0.94,framesep=2mm                     ]
   \verm{Function} \azul{norm} = \verm{power} ( x*x + y*y, 0.5 );
\end{Verbatim}

The {\small\tt deriv} method performs symbolic differentiation :

\begin{Verbatim}[commandchars=\\\{\},formatcom=\small\tt,
   baselinestretch=0.94,framesep=2mm                     ]
   \verm{Function} \azul{norm_x} = norm.deriv ( x );
   \verm{Function} \azul{norm_y} = norm.deriv ( y );
\end{Verbatim}

Functions can also be integrated, see section \numb section 6.


                 %--------------------%
\section{~~\cinza{Fields and functions} [outdated]}\label{\numb section 5.\numb parag 2}
                 %--------------------%

{\normalfont\bfseries The code described in this paragraph is outdated and does not work.}

Consider the mesh built by the code below.

\begin{Verbatim}[commandchars=\\\{\},formatcom=\small\tt,
   baselinestretch=0.94,framesep=2mm                     ]
   Mesh::intended_dimension = 2; // topological dimension
   Mesh::initialize();
   auto & xy = NumericField::multi_dim ("size", 2, "lives on", "points");	
   auto & x = xy[0], & y = xy[1];
   auto & u = NumericField::one_dim ("lives on", "points");
   auto SW = Cell::point ("SW"); x(SW) = -1.1; y(SW) = 0.3;
   auto SE = Cell::point ("SE"); x(SE) = 1; y(SE) = 0;
   auto NE = Cell::point ("NE"); x(NE) = 1; y(NE) = 1;
   auto NW = Cell::point ("NW"); x(NW) = -1; y(NW) = 1;
   auto south ( \verm{tag}::segment, SW.reverse(), SE, \verm{tag}::divided_in, 4, xy );
   auto east  ( \verm{tag}::segment, SE.reverse(), NE, \verm{tag}::divided_in, 2, xy );
   auto north ( \verm{tag}::segment, NE.reverse(), NW, \verm{tag}::divided_in, 4, xy );
   auto west  ( \verm{tag}::segment, NW.reverse(), SW, \verm{tag}::divided_in, 2, xy );
   auto rect_mesh ( \verm{tag}::rectangle, south, east, north, west, xy );
\end{Verbatim}

Note that, besides the {\small\tt NumericField} objects {\small\tt x} and {\small\tt y}
which we have already encountered in paragraph \numb section 1.\numb parag 1,
we introduce another {\small\tt NumericField}, {\small\tt u}, which is meant to hold
the values of the solution of some PDE.

We now declare two functions defined on the mesh, linked to the two fields {\small\tt x} and
{\small\tt y}.
We can declare each function individually, as we did in paragraph \numb section 1.\numb parag 1,
or we can declare the pair and then extract each component :

\begin{Verbatim}[commandchars=\\\{\},formatcom=\small\tt,
   baselinestretch=0.94,framesep=2mm                      ]
   auto & xxyy = FunctionOnMesh::from_field ( xy, "Lagrange degree one" );
   auto & xx = xxyy[0], & yy = xxyy[1];
\end{Verbatim}

These functions are described as {\small\tt "Lagrange degree one"}, which means that they vary
linearly along segments and also inside triangles.
On quadrilaterals, they are polynomials of degree one, meaning they have a linear part plus
a bi-linear one.

We can also declare an unknown function and a test function, to be used in a future variational
problem :

\begin{Verbatim}[commandchars=\\\{\},formatcom=\small\tt,
   baselinestretch=0.94,framesep=2mm                      ]
   auto & uu = FunctionOnMesh::unknown ( u, "Lagrange degree one");
   auto & w = FunctionOnMesh::test ( uu );
\end{Verbatim}

The unknown {\small\tt uu} is related to the {\small\tt NumericField} previously declared,
{\small\tt u}, which provides space to hold, at each vertex of the mesh, a real value.
Hopefully, at the end of the day these will be the values of the solution of our PDE.
The test function does not need to hold any values.
The test function is an abstract object whose only use is to express the
variational formulation.

{\small\tt FunctionOnMesh} objects obey to usual arithmetic rules.
For instance, {\small\tt (xx+yy*w)/uu} is a valid expression in \maniFEM.
They can also be differentiated, like in {\small\tt (xx*yy).deriv(xx)}.
Note that this expression will be evaluated right away, producing the result
{\small\tt yy}, while expressions involving an unknown function or a test function
will produce a delayed derivative object, to be evaluated later.
Thus, {\small\tt w.deriv(xx)} will be evaluated only after replacing the test function
{\small\tt w} by some function in a base of a discretized Hilbert space.

{\small\tt FunctionOnMesh} objects can also be integrated through their method
{\small\tt integrate}, which produces a delayed integral expression.
The integral is not evaluated right away, but only later, with the use of an
{\small\tt Integrator} object (which could be a Gauss quadrature).

Integral expressions are stored as objects belonging to the class\hfil\break
{\small\tt FunctionOnMesh::combinIntegrals}.
These objects can be equated by using their\hfil\break
{\small\tt operator==}, which returns a {VariationalProblem} object.
Thus, the operator\hfil\break
{\small\tt FunctionOnMesh::combinIntegrals::operator==} acts as a factory function
for\hfil\break {\small\tt VariationalProblem} objects.
For instance, {\small\tt x.integrate(rect\_\,mesh) == y.integrate(south)} is a syntactically valid
expression which would produce a {\small\tt VariationalProblem} but actually
gives a run-time error because the {\small\tt operator==} checks that the right hand side
of the variational problem actually contains a function declared as uknown and one
declared as test, and that the left hand side contains a test function but no unknown.
Here is a more meaningful example :

\begin{Verbatim}[commandchars=\\\{\},formatcom=\small\tt,
   baselinestretch=0.94,framesep=2mm                      ]
   auto & var_pb =
       ( uu.deriv(xx)*w.deriv(xx) + uu.deriv(yy)*w.deriv(yy) ) .integrate(rect_mesh)
       == w.integrate(rect_mesh) + w.integrate(south);
\end{Verbatim}

Note how we can mix integrals on different domains; recall that {\small\tt south} is
a side of {\small\tt rect\_\,mesh}.

Dirichlet-type boundary conditions are implemented through the directive
{\small\tt prescribe\_\,on} followed by one or more equalities.

\begin{Verbatim}[commandchars=\\\{\},formatcom=\small\tt,
   baselinestretch=0.94,framesep=2mm                      ] 
   var_pb.prescribe_on (north);  uu == 0.;          w == 0.;
   var_pb.prescribe_on (east);   uu == xx*(1.-yy);  w == 0.;
   var_pb.prescribe_on (west);   uu == 0.;          w == 0.;
\end{Verbatim}

