
          %----------------------------------------------%
\chapter{~~Fields, functions and variational formulations}\label{\numb section 5}
          %----------------------------------------------%


          %--------------------%
\section{~~Fields and functions}\label{\numb section 5.\numb parag 1}
          %--------------------%

As the reader may have already noticed, all examples in {\maniFEM} begin by declaring a
Euclidian {\small\tt \verm{Manifold}} and then go on by building a coordinate system :

\begin{Verbatim}[commandchars=\\\{\},formatcom=\small\tt,
   baselinestretch=0.94,framesep=2mm                     ]
   \verm{Manifold} RR2 ( \textcolor{tag}{tag}::Euclid, \textcolor{tag}{tag}::of_dim, 2 );
   \verm{Function} \azul{xy} = RR2.build_coordinate_system ( \textcolor{tag}{tag}::Lagrange, \textcolor{tag}{tag}::of_degree, 1 );
   \verm{Function} \azul{x} = xy[0], \azul{y} = xy[1];
\end{Verbatim}

See paragraph \numb section 11.\numb parag 3 for more details about tags.

In the above, {\small\tt \verm{xy}} is a {\small\tt \verm{Function}}, a vector field actually,
with two components, {\small\tt x} and {\small\tt y}.
The declaration of {\small\tt xy} starts a complex process; under the curtains
{\maniFEM} declares a {\small\tt \verm{Field}} object associated to {\small\tt xy}.
The {\small\tt \verm{Field}} object changes the behaviour of {\maniFEM} in what regards
initialization of {\small\tt \verm{Cell}}s.
Since {\small\tt xy} is of type Lagrange of degree 1, each newly built vertex
{\small\tt \verm{Cell}} will have memory space reserved for two {\small\tt double}
precision numbers.
If {\small\tt A} is a vertex {\small\tt \verm{Cell}}, an assignment like {\small\tt x(A) = 1.5}
sets the value of the {\small\tt x} component of the {\small\tt \verm{Field}} associated to
{\small\tt xy} at {\small\tt A}.

If we declare {\small\tt xy} to be of type Lagrange of degree 2, not only future vertices will
have space reserved for two {\small\tt double}s, but also future segments.
So, we may assign the value of {\small\tt y} at the middle of segment {\small\tt AB} by using
the syntax {\small\tt y(AB) = 0.75}.

{\small\tt \verm{Function}} objects allow for arithmetic expressions like in

\begin{Verbatim}[commandchars=\\\{\},formatcom=\small\tt,
   baselinestretch=0.94,framesep=2mm                     ]
   \verm{Function} \azul{norm} = \verm{power} ( x*x + y*y, 0.5 );
\end{Verbatim}

The {\small\tt deriv} method performs symbolic differentiation :

\begin{Verbatim}[commandchars=\\\{\},formatcom=\small\tt,
   baselinestretch=0.94,framesep=2mm                     ]
   \verm{Function} \azul{norm_x} = norm.deriv ( x );
   \verm{Function} \azul{norm_y} = norm.deriv ( y );
\end{Verbatim}

Functions can also be integrated, see section \numb section 6.


