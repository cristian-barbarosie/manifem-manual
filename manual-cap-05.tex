
          %--------------------------------------%
\chapter{~~Functions and variational formulations}\label{\numb section 5}
          %--------------------------------------%


          %---------%
\section{~~Functions}\label{\numb section 5.\numb parag 1}
          %---------%

As the reader may have already noticed, all examples in {\maniFEM} begin by declaring a
Euclidian {\small\tt\verm{Manifold}} and then go on by building a coordinate system :

\begin{Verbatim}[commandchars=\\\{\},formatcom=\small\tt,
   baselinestretch=0.94,framesep=2mm                     ]
   \verm{Manifold} RR2 ( \textcolor{tag}{tag}::Euclid, \textcolor{tag}{tag}::of_dim, \laranja{2} );
   \verm{Function} \azul{xy} = RR2 .build_coordinate_system ( \textcolor{tag}{tag}::Lagrange, \textcolor{tag}{tag}::of_degree, \laranja{1} );
   \verm{Function} \azul{x} = xy [\laranja{0}], \azul{y} = xy [\laranja{1}];
\end{Verbatim}

See paragraph \ref{\numb section 11.\numb parag 3} for more details about tags.

In the above, {\small\tt\verm{xy}} is a vector-valued {\small\tt\verm{Function}}
with two components, {\small\tt x} and {\small\tt y}.
The declaration of {\small\tt\azul{xy}} starts a complex process; behind the curtains,
{\maniFEM} builds a {\small\tt\verm{Field}} object associated to {\small\tt xy}.
The {\small\tt\verm{Field}} object changes the behaviour of {\maniFEM} in what regards
initialization of {\small\tt\verm{Cell}}s.
Since {\small\tt xy} is of type Lagrange of degree 1, each newly built vertex
{\small\tt\verm{Cell}} will have memory space reserved for two {\small\tt double}
precision numbers.
If {\small\tt A} is a vertex {\small\tt\verm{Cell}}, an assignment like
{\small\tt x(A) = \laranja{1.5}} sets the value of the {\small\tt x} component of the
{\small\tt\verm{Field}} associated to {\small\tt xy} at {\small\tt A}.
Note that negative vertices will have no coordinates, only positive vertices will.

If we declare {\small\tt xy} to be of type Lagrange of degree 2, not only future vertices will
have space reserved for two {\small\tt double}s, but also future segments.
So, we may assign the value of {\small\tt y} associated to a segment {\small\tt AB} by using
the syntax {\small\tt y(AB) = \laranja{0.75}}.
This number may be interpreted as the value taken by {\small\tt y} at the middle of
segment {\small\tt AB}.
This feature is not yet implemented.

Other {\small\tt\verm{Function}}s are not related with space in the computer's memory
(they have no subjacent {\small\tt\verm{Field}}).
For instance, there are arithmetic expressions like 

\begin{Verbatim}[commandchars=\\\{\},formatcom=\small\tt,baselinestretch=0.94]
   \verm{Function} \azul{norm} = \verm{power} ( x*x + y*y + z*z, \laranja{0.5} );
\end{Verbatim}

We can still access the value of such a {\small\tt\verm{Function}} at a cell like in
{\small\tt double} {\small\tt\azul{n}} {\small\tt =} {\small\tt norm(A)};
the computer will return the value of the corresponding arithmetic expression,
replacing the symbols {\small\tt x}, {\small\tt y} and {\small\tt z} by the corresponding
values at cell {\small\tt A}.
But it makes no sense to change the value of {\small\tt norm} at cell {\small\tt A}.
Thus, statements like {\small\tt x(A)} {\small\tt =} {\small\tt\laranja{1.}} work fine,
while {\small\tt norm(A)} {\small\tt =} {\small\tt\laranja{1.}} produces a run-time error.

The {\small\tt deriv} method performs symbolic differentiation :

\begin{Verbatim}[commandchars=\\\{\},formatcom=\small\tt,
   baselinestretch=0.94,framesep=2mm                     ]
   \verm{Function} \azul{norm_x} = norm .deriv ( x );
   \verm{Function} \azul{norm_y} = norm .deriv ( y );
\end{Verbatim}

Functions can also be integrated, see section \ref{\numb section 6}.


