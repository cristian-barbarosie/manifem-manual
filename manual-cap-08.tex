
          %-----------%
\chapter{~~Miscellanea}\label{\numb section 8}
          %-----------%

This section gives informations which do not fit in other sections of the manual.


          %---------------------------------%
\section{~~Projecting a vertex on a manifold}\label{\numb section 8.\numb parag 1}
          %---------------------------------%

We recall that, in this manual, we use the term ``manifold'' to refer to a manifold without boundary.
Of course we will only mesh a bounded domain of the manifold, so most meshes will have a boundary.
We may also mesh an entire compact manifold like a circle or a sphere or a torus;
such a mesh will have no boundary.

In many examples we consider a submanifold of $ \mathbb{R}^n $ defined by an implicit equation
(or by several implicit equations).
When meshing (a bounded domain of) such a manifold, we often need to project points in
the surrounding space onto our manifold.
This projection operation is achieved by solving
the equation (or system of equations) defining the submanifold.

If the manifold is defined by one or two implicit equations,
these equations are solved numerically by using a Newton-like algorithm.
See e.g.\ Appendix B in [BTL].%
\footnote {{} C.~Barbarosie, A.M.~Toader, S.~Lopes, A gradient-type algorithm for constrained
optimization with application to microstructure optimization, Discrete and Continuous Dynamical
Systems series B, 25, p.\ 1729-1755, 2020}
This algoritm converges only if the initial guess (that is, the coordinates of the point we are
trying to project) is close enough to the manifold.
The user should take this limitation into account.
For two equations, the inverse of the Jacobian matrix is computed by simply applying the definition
(quotients of 2x2 determinants).

If the manifold is defined by three implicit equations (often, this will be a zero-dimensional manifold),
we prefer not to compute the inverse of a 3x3 matrix, so we apply a conjugate gradient method instead.

The case of four or more implicit equations is not implemented.

Many {\small\tt\verm{Mesh}} constructors project vertices on the current working manifold
{\small\tt\verm{Manifold}::working} wihtout the user's assistance.
On the other hand, vertices can be explicitly {\small\tt project}ed if necessary,
either directly within the {\small\tt\verm{Cell}} constructor or by invoking the method
{\small\tt\verm{Mesh}::project}.
Paragraphs \ref{\numb section 2.\numb parag 4} -- \ref{\numb section 2.\numb parag 17},
\ref{\numb section 3.\numb parag 2} -- \ref{\numb section 3.\numb parag 9} show such examples.


          %---------------------%
\section{~~On quotient manifolds}\label{\numb section 8.\numb parag 2}
          %---------------------%

Section \ref{\numb section 7} is devoted to meshes on quotient manifolds.

Usually, a quotient manifold is defined by means of orbits of the ``periodicity'' group.
The manifolds considered in paragraphs \ref{\numb section 7.\numb parag 14} --
\ref{\numb section 7.\numb parag 18} cannot be defined this way if their angle is not a
divisor of $ 2\pi $.
This can be easily circumvented; it suffices to consider only a neighbourhood of a representative
domain instead of the entire $ \mathbb{R}^2 $.
In the case of a singular point (paragraphs \ref{\numb section 7.\numb parag 15} and
\ref{\numb section 7.\numb parag 16}) the above refered neighbourhood should not contain
the singularity.

When we divide a Riemannian manifold (in particular, the Euclidian space $ \mathbb{R}^n $)
by a group of isometries, the resulting (quotient) manifold inherits naturally a Riemann metric.
This is not true for the manifolds considered in paragraphs \ref{\numb section 7.\numb parag 17}
and \ref{\numb section 7.\numb parag 18} -- the actions are not isometries.
In other words, we identify a pair of sides (of the representative domain) having different lengths,
so the quotient operation is not compatible with the metric in $ \mathbb{R}^2 $.
Thus, these quotient manifolds do not inherit a Riemann metric.
They are valid as differentiable manifolds but not as Riemann manifolds.
Length and area are not well-defined in these examples.
Since they are compact, these manifolds can be viewed as uniform spaces.%
\footnote {{} see e.g.\ Corollary 30 in Chapter 6 of J.L.~Kelley, General Topology, Van Nostrad, 1955}


          %--------------------------------------%
\section{~~Exchanging information with {\tt gmsh}}\label{\numb section 8.\numb parag 3}
          %--------------------------------------%

{\small\tt gmsh} is a complex software and does much more than drawing meshes.
In this paragraph we show how {\maniFEM} can take advantage of {\small\tt gmsh}'s capabilities
for writing and reading back complex meshes.

To fix ideas, let us focus on the example in paragraph \ref{\numb section 3.\numb parag 18},
where a mesh called {\small\tt sphere\_\,and\_} {\small\tt\_\,cylinder} is built.
We can save this mesh using method {\small\tt\verm{Mesh}::export\_\,to\_\,file}, then use
{\small\tt gmsh} to draw it.
But if we want to re-use this mesh in another {\maniFEM} program, the declaration

\begin{Verbatim}[commandchars=\\\{\},formatcom=\small\tt,
   baselinestretch=0.94,framesep=2mm                     ]
   \verm{Mesh} \azul{sphere_and_cylinder} ( \textcolor{tag}{tag}::import, \textcolor{tag}{tag}::msh, \verde{"filename.msh"});
\end{Verbatim}

\noindent may be insufficient for our needs.
You see, in the original program (the one presented in paragraph
\ref{\numb section 3.\numb parag 18}) we had at our disposal not only the {\small\tt\verm{Mesh}}
object {\small\tt sphere\_\,and\_\,cylinder} but also other {\tt C++} objects like
{\small\tt piece\_\,of\_\,sph}, {\small\tt piece\_\,of\_\,cyl}, {\small\tt two\_\,circles},
{\small\tt circle\_\,1} and {\small\tt circle\_\,2}.
Depending on our intentions, these objects may be useful.
For instance, we may want to iterate over all segments of {\small\tt circle\_\,1} in order to
impose some boundary condition.
Or, we may want to {\small\tt join} the mesh {\small\tt sphere\_\,and\_\,cylinder} to some
other mesh (say, another sphere).
In order for the {\small\tt join} operation to work well, we must build that other
mesh starting from {\small\tt circle\_\,1} as boundary.
If you don't understand the previous statement, you may want to read again paragraphs
\ref{\numb section 1.\numb parag 4} and \ref{\numb section 3.\numb parag 18}.

The problem is, in the second {\maniFEM} program, the one which {\small\tt import}s the mesh
{\small\tt sphere\_\,and\_} {\small\tt\_\,cylinder} from an {\small\tt msh} file,
we do not have access to the {\small\tt circle\_\,1} object.
True, we can recover it by testing each segment of {\small\tt sphere\_\,and\_\,cylinder}
(or rather, both extremities of each segment) against the equation {\small\tt x} {\small\tt ==}
{\small\tt\laranja{1.5}}, but this not elegant.
It would be nice to extract, from one {\small\tt msh} file, several related
{\small\tt\verm{Mesh}} objects.
And this is only possible if we save, in the first {\maniFEM} program, more precise information
than merely one large {\small\tt\verm{Mesh}} object.

Luckily, the {\small\tt msh} file format allows one to specify regions of a mesh
either through {\small\tt EntityTag}s (which are integer numbers)
or through {\small\tt PhysicalName}s (which are strings).
{\ManiFEM} takes advantage of these features so save (and later recover) several
{\small\tt\verm{Mesh}}es, components of one large {\small\tt\verm{Mesh}}, in (and from)
one {\small\tt msh} file.

In the future, a class {\small\tt\verm{Mesh}::Composed} will be implemented which will
allow the user to save, using a clear and elegant syntax, components of a {\small\tt\verm{Mesh}}
and later retrieve them  from an {\small\tt msh} file.

