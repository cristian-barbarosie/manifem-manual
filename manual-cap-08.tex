
          %-----------------%
\chapter{~~More on manifolds}\label{\numb section 8}
          %-----------------%

This section gives some details on the notion of manifold in \maniFEM.

Let us begin by recalling that, in this manual, we use the term ``manifold'' to refer to
a manifold without boundary.
Of course we will only mesh a bounded domain of the manifold, so the meshes will have a boundary.
We may also mesh an entire compact manifold like a circle or a sphere or a torus;
such a mesh will have no boundary.


          %-----------------------------------------%
\section{~~Projecting points on an implicit manifold}\label{\numb section 8.\numb parag 1}
          %-----------------------------------------%

In many examples we consider a submanifold of $ \mathbb{R}^n $ defined by an implicit equation
(or by several implicit equations).
When meshing (a bounded domain of) such a manifold, we often need to project points in
the surrounding space onto our manifold. This projection operation is achieved by solving
the equation (or system of equations) defining the submanifold.
These equations are solved numerically by using a Newton-like algorithm.
See e.g.\ Appendix B in [BTL].%
\footnote{C.~Barbarosie, A.M.~Toader, S.~Lopes, A gradient-type algorithm for constrained
optimization with application to microstructure optimization, Discrete and Continuous Dynamical
Systems series B, 25, p.\ 1729-1755, 2020}
This algoritm converges only if the initial guess (that is, the coordinates of the point we are
trying to project) is close enough to the manifold.
The user should take this limitation into account.

Many {\small\tt Mesh} constructors project vertices on the current {\small\tt working\_\,manifold}
wihtout the user's assistance.
On the other hand, vertices can be explicitly {\small\tt project}ed if necessary.
Paragraphs \ref{\numb section 2.\numb parag 3} -- \ref{\numb section 2.\numb parag 12},
\ref{\numb section 3.\numb parag 2} -- \ref{\numb section 3.\numb parag 9} show such examples.
