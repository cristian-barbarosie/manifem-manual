
                 %---------------------------------------------------%
\chapter{~~\cinza{Quotient manifolds and periodic boundary conditions}}
                 %---------------------------------------------------%
\label{\numb section 7}


{\normalfont\bfseries The code described in this section does not work yet.
It should be regarded as a mere declaration of intentions.}
\medskip

The roots of {\maniFEM} go back to a PhD thesis in 2002
% (or even earlier, to a Master thesis in 1997)
where finite elements on a torus were implemented in {\small\tt FORTRAN}.%
\footnote{See % C.~Barbarosie, Optimization of perforated domains through homogenization,
% Structural Optimization 14, 1997
C.~Barbarosie, Shape optimization of periodic structures,
Computers \& Structures 30, 2003}
The torus is meant as a mere quotient manifold between $ \mathbb{R}^2 $ and a group of
translations of $ \mathbb{R}^2 $ with two generators;
you may think of it as $ {\mathbb R}^2/{\mathbb Z}^2 $.
It should be stressed that this manifold is not the usual ``donut'' built in paragraph
\ref{\numb section 2.\numb parag 15}.
The quotient torus is a Riemann manifold with no curvature; it is locally Euclidian
(that is, locally isometric to open sets of $ \mathbb{R}^2 $); we may call it ``flat torus''.
It cannot be embedded in $ \mathbb{R}^3 $, much less be represented graphically.
An unfolded mesh in $ \mathbb{R}^2 $ can be represented graphically, where vertices and segments
from the torus are drawn more than once.

One of the goals of {\maniFEM} is to deal with meshes on quotient manifolds.
Different quotient operations can be used, with groups of translations of $ \mathbb{R}^2 $ but also
with other groups of transformations.


                 %--------------------------%
\section{~~\cinza{A circle with no curvature}}\label{\numb section 7.\numb parag 1}
                 %--------------------------%

{\normalfont\bfseries The code described in this paragraph does not work yet.
It should be regarded as a mere declaration of intentions.}

Here is the closed curve $ \mathbb{R}/{\mathbb Z} $.
We define it as a segment from {\small\tt A} to {\small\tt A}, with a specified {\small\tt jump}.

\begin{Verbatim}[commandchars=\\\{\},formatcom=\small\tt,
   baselinestretch=0.94,framesep=2mm                      ]
   \cinza{// begin with the one-dimensional line}
   \verm{Manifold} \azul{RR} ( \verm{tag}::Euclid, \verm{tag}::of_dim, 1 );
   \verm{Function} \azul{x} = RR.build_coordinate_system ( \verm{tag}::Lagrange, \verm{tag}::of_degree, 1 );

   \cinza{// define an action on RR2 (a translation)}
   \verm{Manifold}::Action \azul{g};  g(x) = x+1;
   \verm{Manifold} \azul{circle} = RR.quotient ( g );

   \cinza{// one vertex is enough to start the process}
   \verm{Cell} \azul{A} ( \verm{tag}::vertex );  x(A) = 0.02;
   \cinza{// with this vertex, we build a segment}
   \verm{Mesh} \azul{seg} ( \verm{tag}::segment, A.reverse(), A, \verm{tag}::divided_in, 10, \verm{tag}::jump, g );
\end{Verbatim}

We do not bother with the graphical representation of this one-dimensional mesh.


                 %------------%
\section{~~\cinza{A flat torus}}\label{\numb section 7.\numb parag 2}
                 %------------%

{\normalfont\bfseries The code described in this paragraph does not work yet.
It should be regarded as a mere declaration of intentions.}

Here is the classical example of $ \mathbb{R}^2/{\mathbb Z}^2 $.

\begin{Verbatim}[commandchars=\\\{\},formatcom=\small\tt,
   baselinestretch=0.94,framesep=2mm                      ]
   \cinza{// begin with the usual two-dimensional space}
   \verm{Manifold} \azul{RR2} ( \verm{tag}::Euclid, \verm{tag}::of_dim, 2 );
   \verm{Function} \azul{xy} = RR2.build_coordinate_system ( \verm{tag}::Lagrange, \verm{tag}::of_degree, 1 );
   \verm{Function} \azul{x} = xy[0], \azul{y} = xy[1];

   \cinza{// define two actions on RR2 (translations)}
   \verm{Manifold}::Action \azul{g1}, \azul{g2};
   g1(x,y) = (x+1) && y;
   g2(x,y) = x && (y+1);
   \verm{Manifold} \azul{torus_manif} = RR2.quotient ( g1, g2 );

   \cinza{// one vertex is enough to start the process}
   \verm{Cell} \azul{A} ( \verm{tag}::vertex );  x(A) = 0.02;  y(A) = 0.02;
   \cinza{// with this vertex, we build two segments}
   \verm{Mesh} \azul{seg_horiz} ( \verm{tag}::segment, A.reverse(), A,
                    \verm{tag}::divided_in, 10, \verm{tag}::jump, g1 );
   \verm{Mesh} \azul{seg_vert}  ( \verm{tag}::segment, A.reverse(), A,
                    \verm{tag}::divided_in, 10, \verm{tag}::jump, g2 );
   \cinza{// and a rectangle}
   \verm{Mesh} \azul{torus} ( \verm{tag}::rectangle, seg_horiz, seg_vert,
                seg_horiz.reverse(), seg_vert.reverse() );

   \cinza{// it would be meaningless to export 'square' as a msh file}
   \cinza{// we can however export an unfolded mesh :}
   torus.unfold(-0.5,-0.2,0.5,0.2).export_msh (\verde{"unfolded-torus.msh"});
\end{Verbatim}

We have added a shadow representing the periodicity cell $ [0,1]^2 $.
This gives a hint about the segments being repeated by the unfolding.


                 %-----------------%
\section{~~\cinza{A skew flat torus}}\label{\numb section 7.\numb parag 3}
                 %-----------------%

{\normalfont\bfseries The code described in this paragraph does not work yet.
It should be regarded as a mere declaration of intentions.}

We can build a skew torus by simply choosing other actions on {\small\tt RR2}.

\begin{Verbatim}[commandchars=\\\{\},formatcom=\small\tt,
   baselinestretch=0.94,framesep=2mm                      ]
   \verm{Manifold}::Action \azul{g1}, \azul{g2};
   g1(x,y) = (x+1)   && (y+0.1);
   g2(x,y) = (x+0.1) && (y+1);
   \verm{Manifold} \azul{torus_manif} = RR2.quotient ( g1, g2 );
\end{Verbatim}

Again, we have added a shadow representing the periodicity cell, this time a parallelogram.


                 %---------------%
\section{~~\cinza{A curved circle}}\label{\numb section 7.\numb parag 4}
                 %---------------%

{\normalfont\bfseries The code described in this paragraph does not work yet.
It should be regarded as a mere declaration of intentions.}

We are now in a position to resume the example in paragraph \ref{\numb section 2.\numb parag 14},
this time in a less cumbersome manner.

\begin{Verbatim}[commandchars=\\\{\},formatcom=\small\tt,
   baselinestretch=0.94,framesep=2mm                      ]
   \verm{Manifold} \azul{RR} ( \verm{tag}::Euclid, \verm{tag}::of_dim, 1 );
   \verm{Function} \azul{theta} = RR.build_coordinate_system ( \verm{tag}::Lagrange, \verm{tag}::of_degree, 1 );
   const double \azul{pi} = 3.1415926536;
   \verm{Manifold}::Action \azul{g};  g(theta) = theta + 2*pi;
   \verm{Manifold} \azul{circle} = RR.quotient ( g );

   \verm{Cell} \azul{A} ( \verm{tag}::vertex );  theta(A) = 0.;
   \verm{Mesh} \azul{seg} ( \verm{tag}::segment, A.reverse(), A, \verm{tag}::divided_in, 10, \verm{tag}::jump, g );

   \cinza{// define new coordinates x and y as arithmetic expressions of theta}
   \verm{Function} \azul{x} = \verm{cos}(theta), \azul{y} = \verm{sin}(theta);
   \cinza{// forget about theta; in future statements, x and y will be used}
   circle.set_coordinates ( x && y );
   seg.export_msh (\verde{"circle.msh"});
\end{Verbatim}


                 %----------%
\section{~~\cinza{A cylinder}}\label{\numb section 7.\numb parag 5}
                 %----------%

{\normalfont\bfseries The code described in this paragraph does not work yet.
It should be regarded as a mere declaration of intentions.}

Here is how to build a cylinder in $ \mathbb{R}^3 $ :

\begin{Verbatim}[commandchars=\\\{\},formatcom=\small\tt,
   baselinestretch=0.94,framesep=2mm                      ]
   \verm{Manifold} \azul{RR2} ( \verm{tag}::Euclid, \verm{tag}::of_dim, 2 );
   \verm{Function} \azul{theta_z} =
      RR2.build_coordinate_system ( \verm{tag}::Lagrange, \verm{tag}::of_degree, 1 );
   \verm{Function} \azul{theta} = theta_z[0], \azul{z} = theta_z[1];
   const double \azul{pi} = 3.1415926536;
   \verm{Manifold}::Action \azul{g};  g ( theta, z ) = (theta+2*pi) && z;
   \verm{Manifold} \azul{cylinder_manif} = RR2.quotient ( g );

   \verm{Cell} \azul{A} ( \verm{tag}::vertex );  theta(A) = 0.; z(A) = -1.;
   \verm{Cell} \azul{B} ( \verm{tag}::vertex );  theta(B) = 0.; z(B) =  1.;
   \verm{Mesh} \azul{AA} ( \verm{tag}::segment, A.reverse(), A, \verm{tag}::divided_in, 10, \verm{tag}::jump,  g );
   \verm{Mesh} \azul{AB} ( \verm{tag}::segment, A.reverse(), B, \verm{tag}::divided_in, 10 );  \cinza{// no jump}
   \verm{Mesh} \azul{BB} ( \verm{tag}::segment, B.reverse(), B, \verm{tag}::divided_in, 10, \verm{tag}::jump, -g );
   \verm{Mesh} \azul{cylinder} ( \verm{tag}::rectangle, AA, AB, BB, AB.reverse() );

   \cinza{// define new coordinates x and y as arithmetic expressions of theta}
   \verm{Function} \azul{x} = \verm{cos}(theta), \azul{y} = \verm{sin}(theta);
   \cinza{// forget about theta; in future statements, x, y and z will be used}
   cylinder_manif.set_coordinates ( x && y & z );
   cylinder.export_msh (\verde{"circle.msh"});
\end{Verbatim}


                 %--------------%
\section{~~\cinza{A curved torus}}\label{\numb section 7.\numb parag 6}
                 %--------------%

{\normalfont\bfseries The code described in this paragraph does not work yet.
It should be regarded as a mere declaration of intentions.}

We are now in a position to resume the example in paragraph \ref{\numb section 2.\numb parag 15},
this time by using the quotient manifold $ \mathbb{R}^2/{\mathbb Z}^2 $ introduced in paragraph
\ref{\numb section 7.\numb parag 2}.

\begin{Verbatim}[commandchars=\\\{\},formatcom=\small\tt,
   baselinestretch=0.94,framesep=2mm                      ]
   \verm{Manifold} \azul{RR2} ( \verm{tag}::Euclid, \verm{tag}::of_dim, 2 );
   \verm{Function} \azul{ab} = RR2.build_coordinate_system ( \verm{tag}::Lagrange, \verm{tag}::of_degree, 1 );
   \verm{Function} \azul{alpha} = ab[0], \azul{beta} = ab[1];
   const double \azul{pi} = 3.1415926536;
   \verm{Manifold}::Action \azul{g1}, \azul{g2};
   g1(alpha,beta) = (alpha+2.*pi) && beta;
   g2(alpha,beta) = alpha && (beta+2.*pi);
   \verm{Manifold} \azul{torus_manif} = RR2.quotient ( g1, g2 );

   \verm{Cell} \azul{A} ( \verm{tag}::vertex );  alpha(A) = 0.;  beta(A) = 0.;
   \verm{Mesh} \azul{seg_horiz} ( \verm{tag}::segment, A.reverse(), A,
                    \verm{tag}::divided_in, 10, \verm{tag}::jump, g1 );
   \verm{Mesh} \azul{seg_vert}  ( \verm{tag}::segment, A.reverse(), A,
                    \verm{tag}::divided_in, 10, \verm{tag}::jump, g2 );
   \verm{Mesh} \azul{torus} ( \verm{tag}::rectangle, seg_horiz, seg_vert,
                seg_horiz.reverse(), seg_vert.reverse() );

   \cinza{// parametrize the donut}
   const double \azul{big_radius} = 3., \azul{small_radius} = 1.;
   \cinza{// define x, y and z as functions of alpha and beta}
   \verm{Function} \azul{x} = ( big_radius + small_radius*\verm{cos}(beta) ) * \verm{cos}(alpha),
            \azul{y} = ( big_radius + small_radius*\verm{cos}(beta) ) * \verm{sin}(alpha),
            \azul{z} = small_radius*\verm{sin}(beta);

   \cinza{// forget about alpha and beta :}
   torus.set_coordinates ( x && y && z );
   \cinza{// in future statements (e.g. for graphical representation)}
   \cinza{// x, y and z will be used, not alpha nor beta :}
   torus.export_msh (\verde{"torus.msh"});
\end{Verbatim}


